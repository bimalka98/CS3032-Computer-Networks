\documentclass[a4paper,11pt]{article}

%% packages

\usepackage{blindtext} % needed for creating dummy text passages
%\usepackage{ngerman} % needed for German default language
\usepackage{amsmath} % needed for command eqref
\usepackage{amssymb} % needed for math fonts
\usepackage[colorlinks=true,breaklinks]{hyperref} % needed for creating hyperlinks in the document, the option colorlinks=true gets rid of the awful boxes, breaklinks breaks lonkg links (list of figures), and ngerman sets everything for german as default hyperlinks language
\usepackage[hyphenbreaks]{breakurl} % ben�tigt f�r das Brechen von URLs in Literaturreferenzen, hyphenbreaks auch bei links, die �ber eine Seite gehen (mit hyphenation).
\usepackage{xcolor}
\definecolor{c1}{rgb}{0,0,1} % blue
\definecolor{c2}{rgb}{0,0.3,0.9} % light blue
\definecolor{c3}{rgb}{0.3,0,0.9} % red blue
\hypersetup{
    linkcolor={c1}, % internal links
    citecolor={c2}, % citations
    urlcolor={c3} % external links/urls
}
%\usepackage{cite} % needed for cite
\usepackage[square,authoryear]{natbib} % needed for cite and abbrvnat bibliography style
\usepackage[nottoc]{tocbibind} % needed for displaying bibliography and other in the table of contents
\usepackage{graphicx} % needed for \includegraphics 
\usepackage{longtable} % needed for long tables over pages
\usepackage{bigstrut} % needed for the command \bigstrut
\usepackage{enumerate} % needed for some options in enumerate
%\usepackage{todonotes} % needed for todos
\usepackage{makeidx} % needed for creating an index
\makeindex
\usepackage{gensymb}
\usepackage{url}

%% page settings

\usepackage[top=5mm, bottom=5mm,left=15mm,right=15mm]{geometry} % needed for page border settings
\parindent=0mm % for space of first line of new text block
\sloppy % for writing with hyphenless justification (tries to)
\hyphenation{} % use hyphenation of tolerance parametershttp://www.jr-x.de/publikationen/latex/tipps/zeilenumbruch.html
\hyphenpenalty=10000
\exhyphenpenalty=10000
\usepackage{fancyhdr} % needed for head and foot options
%% my macros

%% Text fomats
\newcommand{\tbi}[1]{\textbf{\textit{#1}}}

%% Math fonts
\newcommand{\bbA}{\mathbb{A}}
\newcommand{\bbB}{\mathbb{B}}
\newcommand{\bbC}{\mathbb{C}}
\newcommand{\bbD}{\mathbb{D}}
\newcommand{\bbE}{\mathbb{E}}
\newcommand{\bbF}{\mathbb{F}}
\newcommand{\bbG}{\mathbb{G}}
\newcommand{\bbH}{\mathbb{H}}
\newcommand{\bbI}{\mathbb{I}}
\newcommand{\bbJ}{\mathbb{J}}
\newcommand{\bbK}{\mathbb{K}}
\newcommand{\bbL}{\mathbb{L}}
\newcommand{\bbM}{\mathbb{M}}
\newcommand{\bbN}{\mathbb{N}}
\newcommand{\bbO}{\mathbb{O}}
\newcommand{\bbP}{\mathbb{P}}
\newcommand{\bbQ}{\mathbb{Q}}
\newcommand{\bbR}{\mathbb{R}}
\newcommand{\bbS}{\mathbb{S}}
\newcommand{\bbT}{\mathbb{T}}
\newcommand{\bbU}{\mathbb{U}}
\newcommand{\bbV}{\mathbb{V}}
\newcommand{\bbW}{\mathbb{W}}
\newcommand{\bbX}{\mathbb{X}}
\newcommand{\bbY}{\mathbb{Y}}
\newcommand{\bbZ}{\mathbb{Z}}

\usepackage{tikz}
\usetikzlibrary{shapes,arrows}

\begin{document}

	Thalagala B.P. 180631J
\begin{center}
{	\Large\textbf{Home Work : Transport Layer}}\\[2mm]

\textbf{December 15, 2021}
\end{center}


\subsection*{1. Why are transport layer services called end-to-end?}

Transport layer provides a point-to-point reliable connection between two end stations. 

\subsection*{2. How do transport layer services differ from network layer services?}


\subsection*{3. In what way are transport layer services similar with data link layer services?}


\subsection*{4. What are the types of transport services?}


\subsection*{5. What is the theoretical maximum number of Internet connection?    Note: IPv4 address and a port together represent the transport address}


\subsection*{6. What are the transport layer primitives? How are they related with connection-less service?}


\subsection*{7. How is a connection established before starting a communication? It may not happen at once. What are the possible issues that can exist? How is it solved?}


\subsection*{8. Draw diagrams for following scenarios (you may use a diagramming tool and include an image in the document, or you may draw on a piece of paper, take a photo and include it in the document)}

\subsubsection*{8.1. Connection establishment}

\begin{itemize}
\item Normal case of a three-way handshake
\item Old CONNECTION REQUEST appearing out of nowhere
\item Duplicate CONNECTION REQUEST  and duplicate ACK
\end{itemize}

\subsubsection*{8.2. Connection Release}

\begin{itemize}
\item Normal case of a three-way handshake
\item Final ACK lost
\item Response lost
\item Response lost and subsequent DRs lost
\end{itemize}

\subsection*{9. What is the objective of having flow control mechanism?}

\subsection*{10. How does buffer help controlling the flow?}

\end{document}
