\documentclass[a4paper,11pt]{article}

\input{settings/packages}
\input{settings/page}
\input{settings/macros}

\usepackage{tikz}
\usetikzlibrary{shapes,arrows}

\begin{document}

	Thalagala B.P. 180631J
\begin{center}
{	\Large\textbf{Home Work : Transport Layer}}\\[2mm]

\textbf{December 15, 2021}
\end{center}


\subsection*{1. Why are transport layer services called end-to-end?}

Transport layer provides a point-to-point reliable connection between two end stations. 

\subsection*{2. How do transport layer services differ from network layer services?}


\subsection*{3. In what way are transport layer services similar with data link layer services?}


\subsection*{4. What are the types of transport services?}


\subsection*{5. What is the theoretical maximum number of Internet connection?    Note: IPv4 address and a port together represent the transport address}


\subsection*{6. What are the transport layer primitives? How are they related with connection-less service?}


\subsection*{7. How is a connection established before starting a communication? It may not happen at once. What are the possible issues that can exist? How is it solved?}


\subsection*{8. Draw diagrams for following scenarios (you may use a diagramming tool and include an image in the document, or you may draw on a piece of paper, take a photo and include it in the document)}

\subsubsection*{8.1. Connection establishment}

\begin{itemize}
\item Normal case of a three-way handshake
\item Old CONNECTION REQUEST appearing out of nowhere
\item Duplicate CONNECTION REQUEST  and duplicate ACK
\end{itemize}

\subsubsection*{8.2. Connection Release}

\begin{itemize}
\item Normal case of a three-way handshake
\item Final ACK lost
\item Response lost
\item Response lost and subsequent DRs lost
\end{itemize}

\subsection*{9. What is the objective of having flow control mechanism?}

\subsection*{10. How does buffer help controlling the flow?}

\end{document}
