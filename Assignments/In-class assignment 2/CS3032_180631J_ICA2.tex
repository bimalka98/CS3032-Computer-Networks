\documentclass[a4paper,11pt]{article}

%% packages

\usepackage{blindtext} % needed for creating dummy text passages
%\usepackage{ngerman} % needed for German default language
\usepackage{amsmath} % needed for command eqref
\usepackage{amssymb} % needed for math fonts
\usepackage[colorlinks=true,breaklinks]{hyperref} % needed for creating hyperlinks in the document, the option colorlinks=true gets rid of the awful boxes, breaklinks breaks lonkg links (list of figures), and ngerman sets everything for german as default hyperlinks language
\usepackage[hyphenbreaks]{breakurl} % ben�tigt f�r das Brechen von URLs in Literaturreferenzen, hyphenbreaks auch bei links, die �ber eine Seite gehen (mit hyphenation).
\usepackage{xcolor}
\definecolor{c1}{rgb}{0,0,1} % blue
\definecolor{c2}{rgb}{0,0.3,0.9} % light blue
\definecolor{c3}{rgb}{0.3,0,0.9} % red blue
\hypersetup{
    linkcolor={c1}, % internal links
    citecolor={c2}, % citations
    urlcolor={c3} % external links/urls
}
%\usepackage{cite} % needed for cite
\usepackage[square,authoryear]{natbib} % needed for cite and abbrvnat bibliography style
\usepackage[nottoc]{tocbibind} % needed for displaying bibliography and other in the table of contents
\usepackage{graphicx} % needed for \includegraphics 
\usepackage{longtable} % needed for long tables over pages
\usepackage{bigstrut} % needed for the command \bigstrut
\usepackage{enumerate} % needed for some options in enumerate
%\usepackage{todonotes} % needed for todos
\usepackage{makeidx} % needed for creating an index
\makeindex
\usepackage{gensymb}
\usepackage{url}

%% page settings

\usepackage[top=5mm, bottom=5mm,left=15mm,right=15mm]{geometry} % needed for page border settings
\parindent=0mm % for space of first line of new text block
\sloppy % for writing with hyphenless justification (tries to)
\hyphenation{} % use hyphenation of tolerance parametershttp://www.jr-x.de/publikationen/latex/tipps/zeilenumbruch.html
\hyphenpenalty=10000
\exhyphenpenalty=10000
\usepackage{fancyhdr} % needed for head and foot options
%% my macros

%% Text fomats
\newcommand{\tbi}[1]{\textbf{\textit{#1}}}

%% Math fonts
\newcommand{\bbA}{\mathbb{A}}
\newcommand{\bbB}{\mathbb{B}}
\newcommand{\bbC}{\mathbb{C}}
\newcommand{\bbD}{\mathbb{D}}
\newcommand{\bbE}{\mathbb{E}}
\newcommand{\bbF}{\mathbb{F}}
\newcommand{\bbG}{\mathbb{G}}
\newcommand{\bbH}{\mathbb{H}}
\newcommand{\bbI}{\mathbb{I}}
\newcommand{\bbJ}{\mathbb{J}}
\newcommand{\bbK}{\mathbb{K}}
\newcommand{\bbL}{\mathbb{L}}
\newcommand{\bbM}{\mathbb{M}}
\newcommand{\bbN}{\mathbb{N}}
\newcommand{\bbO}{\mathbb{O}}
\newcommand{\bbP}{\mathbb{P}}
\newcommand{\bbQ}{\mathbb{Q}}
\newcommand{\bbR}{\mathbb{R}}
\newcommand{\bbS}{\mathbb{S}}
\newcommand{\bbT}{\mathbb{T}}
\newcommand{\bbU}{\mathbb{U}}
\newcommand{\bbV}{\mathbb{V}}
\newcommand{\bbW}{\mathbb{W}}
\newcommand{\bbX}{\mathbb{X}}
\newcommand{\bbY}{\mathbb{Y}}
\newcommand{\bbZ}{\mathbb{Z}}

\usepackage{tikz}
\usetikzlibrary{shapes,arrows}

\begin{document}
	
	
% Define block styles
\tikzstyle{decision} = [diamond, draw, fill=blue!20, 
text width=4.5em, text badly centered, node distance=3cm, inner sep=0pt]
\tikzstyle{block} = [rectangle, draw, fill=blue!20, 
text width=8em, text centered, rounded corners, minimum height=4em]
\tikzstyle{line} = [draw, -latex']
\tikzstyle{cloud} = [draw, ellipse,fill=red!20, node distance=3cm,
minimum height=4em]
	
	Thalagala B.P. 180631J
\begin{center}	
{	\Large\textbf{In-class assignment 2 : Back-off Algorithm}}\\[2mm]	

\textbf{November 3, 2021}
\end{center}

%4.2.3.2.4 Collision detection and enforcement (half duplex mode only)
% page 136

%In full duplex mode, there is never contention for a shared physical medium. The Physical Layer may
%indicate to the MAC that there are simultaneous transmissions by both stations, but since these transmissions
%do not interfere with each other, a MAC operating in full duplex mode must not react to such Physical Layer
%indications. Full duplex stations do not defer to received traffic, nor abort transmission, jam, backoff, and
%reschedule transmissions as part of Transmit Media Access Management. Transmissions may be initiated
%whenever the station has a packet queued, subject only to the interpacket gap required to allow recovery for
%other sublayers and for the physical medium

In half duplex mode, a collision occurs when two stations try to transmit packets simultaneously. When a collision is detected, instead of terminating the transmission immediately, transmitter continues to send a set of additional bits called as \textit{collision enforcement bits} or \textit{jam bits} of length{\tt jamSize}. This ensures that the collision is detected by all the transmitting stations on the network. When a transmission attempt has been terminated due to a collision, transmitter retries until either it is successful or a maximum number of attempts({\tt attemptLimit = 16}) have been made and all have terminated due to collisions.\\ 

These retransmissions are scheduled by a randomization process called \textit{truncated binary exponential backoff}. The delay, before attempting a retransmission is an integer multiple of time taken to emits 512 bits, called as slot time({\tt slotTime}). The number of slot times to delay before the $n^{th}$ retransmission attempt is chosen as a uniformly distributed random integer $r$ in the range $0\leq r < 2^k$ where $k = min(n, 10)$. After 16 re-tries({\tt attemptLimit}) this event is reported as an error to the higher layers and interface gives-up the transmission.\\

{\small Note: In the below diagram Transmission is abbreviated as TX; ``Deferring On?'' decision block delays the packet for Inter Frame Gap(IFG) once MAC identifies the channel is idle and available for a transmission.}
\vfill

\begin{figure}[!h]
	\centering
	\begin{tikzpicture}[node distance = 2.4cm, auto]
	% Place nodes
	\node [block] (init) {Packet Ready};
	\node [decision, below of=init] (dec1) {Deferring On?};
	\node [block, below of=dec1] (process1) {Start Transmission};
	\node [decision, below of=process1] (dec2) {Collision Detected?};
	\node [decision, below of=dec2] (dec3) {TX Done?};
	\node [cloud, below of=dec3] (end1) {TX Successful};
	
	\node [block, right of= dec1, xshift=6cm, yshift=-1cm] (process2) {Send Jamming Signal of {\tt jamSize}};
	\node [block, below of=process2] (process3) {Increment attempts};	
	\node [decision, below of=process3] (dec4) {Attempts $>$ 16};
	\node [cloud, left of=dec4] (end2) {Error};
	\path [line] (dec4) -- node  {Yes} (end2);
	
	\node [block, below of=dec4] (process4) {Compute Backoff {\tt slotTime*r}};	
	\node [block, below of=process4] (process5) {Wait for backoff time};	
		
	% Draw edges
	\path [line] (init) -- (dec1);
	\path [line] (dec1) -- node  {No} (process1);
	%\path [line] (dec1) -- +(-2,0)  -- +(-2,1.8) --+(0,1.8) ;
	\path [line] (dec1) -- +(-2,0) |- node[near start] {Yes} (0,-1.2) ;
	
	\path [line] (process1) -- (dec2);
	\path [line] (dec2) -- node  {No} (dec3);
	\path [line] (dec3) -- node  {Yes} (end1);
	\path [line] (dec3) -- +(-2,0) |- node[near start] {No} (dec2) ;
	\path [line] (process2) -- (process3);
	\path [line] (process3) -- (dec4);
	\path [line] (dec4) -- node  {No} (process4);
	\path [line] (process4) -- (process5);
	
	
	\draw [line] (dec2) -- +(4,0) |- node[near start] {yes} (process2);	
	\draw [line](process5) -- +(0,-1.5) -- +(3,-1.5) |- (0,-1.2);
	
\end{tikzpicture}
\caption{Block-level circuit diagram of Backoff algorithm}
\end{figure}

{\scriptsize \textbf{References}: Ieee standard for ethernet. \textit{IEEE Std 802.3-2018 (Revision of IEEE Std 802.3-2015)}; Minimizing Mobiles Communication Time Using Modified Binary Exponential Backoff Algorithm: http://dx.doi.org/10.5121/ijcnc.2013.5605  }


\end{document}